
\documentclass{beamer}	
\mode<presentation>
 
\usepackage{pdfpages}
\usepackage{fancyvrb}
\usepackage{chemarr}

\usepackage{amsmath}		%% mathematics typesetting
\usepackage{amssymb}
 
\usepackage{epigraph}   %% nice setting of quotations

\usepackage{tabularx} %% allows to use row colours in tables

\usepackage{ulem}

\usepackage{booktabs}

\usepackage{siunitx} %% tpyeset SI units

\usepackage{CJKutf8} %% typeset Chinese characters

\usepackage{pdfpages}%% include pdfs

\usepackage{graphicx}
\usepackage{animate} %% show animated gifs

\DeclareMathAlphabet{\mathcalligra}{T1}{calligra}{m}{n}


% Color and Theme. Can be changed. However, this one's quite nice.
\usetheme{Madrid}
\definecolor{theme}{rgb}{0.84,0,0.21}
\usecolortheme[named=theme]{structure}

%%  Title information
\title[M01.9 Repetitorium Physik]{M01.9 Repetitorium Physik \\ Teil 2}
\author[melanie.stefan@medicalschool-berlin.de]{}
\institute[]{Prof. Ervice Pouokam Kamgne, Prof. Melanie Stefan \\ melanie.stefan@medcialschool-berlin.de}
\date{SoSe 2022}
 

% Table of contents to pop up at the beginning of each section
\AtBeginSection[]
{
  \begin{frame}<beamer>
    \frametitle{Outline}
    \tableofcontents[currentsection,currentsubsection]
  \end{frame}
}
 
\beamertemplatenavigationsymbolsempty

\begin{document}


{ \usebackgroundtemplate{\includegraphics[width=1.2\paperwidth]{MSB_Titelseite.pdf}} 
\begin{frame}

 \maketitle 

$\,$\\[6cm] 
\end{frame}
}

\section{M01.1 Methodik, Grundbegriffe, Fehlerrechnung}

% %% Würfel

% Würfelförmige Zellen mit jeweils \(10\,\mu m\) Kantenlänge  bilden mit vernachlässigbar kleinen Zwischenräumen ein Gewebe.

% Wie viele Zellen sind in \(1\,cm^3 \) eines derartigen Gewebes enthalten?

% \begin{decription}
% \item[A]
% 1000
% \item[B]
% 1 Million
% \item[C]
% 1 Milliarde
% \item[D]
% 100 Milliarden
% \item[E]
% 1 Billion
% \end{decription}



% %% Messfehler

% Die Messunsicherheit eines Blutdruckmessgeräts wird in der Bedienungsanleitung mit \(\pm 3\, mmHg\) angegeben. Bei einer Messung wird als diastolischer Wert \(90\,mmHg\) angezeigt.

% Etwa wie groß ist somit die relative Mssunsicherheit dieses Wertes (wenn die Angabe in der Bedieungsanleitung zutrifft)?


% \begin{decription}
% \item[A]
% \(\pm 0,2\,\%\) 
% \item[B]
% \(\pm 0,3\,\%\) 
% \item[C] 
% \(\pm 0,9\,\% \) 
% \item[D]
% \(\pm 2\,\%\) 
% \item[E]
% \(\pm 3\,\%\) 
% \end{decription}



% %% Männer

% Bei einer großen Anzahl männlicher Probanden wurde eine Reihenuntersuchung durchgeführt. Es ergab sich für die Körpergröße eine (Gauß-) Normalverteiling mit einem Mittelwert von \(1,80\,m\) und einer Standardabwichung von \(10\,cm\) 

% Welche Aussage zur Größenverteilung dieser Männer trifft am ehesten zu? 
% \begin{decription}
% \item[A]
% Etwa \(5\,\%\) sind mindestens \(2\,m\) groß
% \item[B]
% Etwa \(16\,\%\) sind höchstens \(1,50 \, m\) groß
% \item[C] 
% Etwa \(32\,\% \) sind zwischen \(1,70\,m\) und \(1,90\,m\) groß

% \item[D]
% Etwa \(50\,\%\) sind mindestens \(1,80\,m\) groß
% \item[E]
% Etwa  \(68\,\%\) sind höchstens \(1,90\,m\) groß
% \end{decription}





%% Irgendwas mit Fehlern?


\section{M01.2 Mechanik}

\section{M01.3 Aerodynamik Hydrodynamik, Viskosität und Grenzflächeneffekte}


\section{M01.5 Schwingungen und Wellen}





%% %% %% %% Feedbackhinweisblock

% \begin{frame}
% \frametitle{Danke für Ihr Feedback!}

% \begin{columns}[c]

% \begin{column}{6cm}
% \begin{center}
%  \includegraphics[width=\textwidth]{smilie_balloons.jpg}
% \end{center}

% \end{column}

% \begin{column}{4cm}


% \begin{center}
% \includegraphics[width=\textwidth]{feedback_QR.png}
% \end{center}
% \end{column}


% \end{columns}
% \end{frame}


%% %% %% Bildnachweis
\begin{frame}
\frametitle{Bildnachweis}
\begin{tiny}

\begin{itemize}



% %% all lectures
% \item
% Luftballons mit frohen und traurigen Smilies. Photo by \href{https://unsplash.com/@artbyhybrid?utm_source=unsplash&utm_medium=referral&utm_content=creditCopyText}{Hybrid} on \href{https://unsplash.com/s/photos/feedback?utm_source=unsplash&utm_medium=referral&utm_content=creditCopyText}{Unsplash}
% %%%%%%%%%%%



\end{itemize}
\end{tiny}
\end{frame}






\end{document}

%%% Frequently used snippets

%% \begin{columns}[c]

%% \begin{column}{5cm}
%% \end{column}

%% \begin{column}{5cm}
%% \end{column}


%% \end{columns}




